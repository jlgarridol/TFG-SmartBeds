\apendice{Especificación de Requisitos}

\section{Introducción}

El primer paso de todo desarrollo de software es definir bien los requisitos de la aplicación así como los detalles generales que se necesitan de la aplicación. En este apartado se desarrollaran los requisitos y los casos de uso.

\section{Objetivos generales}

La aplicación web tiene lo siguientes objetivos:
\begin{itemize}
	\tightlist
	\item
	Monitorizar en tiempo real a los pacientes con epilepsia.
	\item 
	Gestionar los distintos accesos del personal médico a los datos de los pacientes.
	\item 
	Ofrecer una API REST de la aplicación.
	\item 
	Crear una interfaz web fácil e intuitiva.
\end{itemize}

\section{Catálogo de requisitos}
En esta sección se presentan los requisitos funcionales y los no funcionales. Tanto los requisitos funcionales como su especificación de los mismos en la sección~\ref{casos-uso} son comunes a los de Alicia Olivares Gil debido a que la aplicación desarrollada está adscrita al mismo proyecto de investigación y la base de la aplicación a realizar es la misma.
\subsection{Requisitos funcionales}\label{requisitos-funcionales}
\begin{itemize}
\tightlist
\item
\textbf{RF-1 Confidencialidad del sistema:} solamente usuarios autorizados podrán acceder al sistema.
	\begin{itemize}
		\tightlist
		\item
		\textbf{RF-1.1 Identificación de usuario:} los usuarios se identificarán con un \textit{nickname} y una contraseña 
		\item
		\textbf{RF-1.2 Rol de administración:} existirá un usuario especial que podrá administrar el sistema completamente sin restricciones.
		\item 
		\textbf{RF-1.3 Visualización de una cama:} los usuarios validados deben poder observar los datos en tiempo real de las camas disponibles. 
		\item
		\textbf{RF-1.4 Restricción de acceso:} los usuarios solamente podrán tener acceso a los datos de las camas permitidas. 
		\item
		\textbf{RF-1.5 Acceso completo al administrador:} el administrador debe poder acceder a todas las camas existentes. 
	\end{itemize}	
\item
\textbf{RF-2 Gestión de las camas:} el administrador ha de gestionar las camas pudiendo añadir, modificar y borrar.
	\begin{itemize}
		\tightlist
		\item
		\textbf{RF-2.1 Añadir cama:} el administrador ha de poder añadir una nueva cama al sistema.
		\item
		\textbf{RF-2.2 Modificar cama:} el administrador ha de poder modificar los datos una cama existente.
		\item
		\textbf{RF-2.3 Borrar cama:} el administrador ha de poder borrar una cama del sistema.
		\item
		\textbf{RF-2.4 Asignar camas a usuarios:} el administrador se encarga de decidir que usuario puede acceder a que cama.
	\end{itemize}
\item
\textbf{RF-3 Gestión de los usuarios:} el administrador ha de gestionar los usuarios pudiendo añadir, modificar y borrar.
	\begin{itemize}
		\tightlist
		\item
		\textbf{RF-3.1 Añadir usuario:} el administrador ha de poder añadir un nuevo usuario al sistema.
		\item
		\textbf{RF-3.2 Modificar usuario:} el administrador ha de poder modificar los datos un usuario existente.
		\item
		\textbf{RF-3.3 Borrar usuario:} el administrador ha de poder borrar un usuario del sistema.
	\end{itemize}
\item 
\textbf{RF-4 Visualización de los datos:} los usuarios han de poder ver de las camas disponibles el estado actual del paciente, sus constantes vitales y las presiones.
\end{itemize}	
\subsection{Requisitos no funcionales}\label{requisitos-no-funcionales}
\begin{itemize}
\tightlist
\item
\textbf{RNF-1 Usabilidad:} la aplicación debe cumplir estándares de usabilidad teniendo una curva de aprendizaje baja y un uso de metáforas adecuado.
\item 
\textbf{RNF-2 Disponibilidad:} las camas existentes han de ser siempre accesibles por sus usuarios asociados y dar una información correcta de su estado
\item 
\textbf{RNF-3 Confidencialidad:} los datos de las camas, al ser en parte constantes vitales de pacientes, solamente han de ser accesibles por los usuarios autorizados.
\item
\textbf{RNF-4 Escalabilidad:} el sistema debe ser escalable para adaptarse mejor a un incremento de carga del sistema.
\item 
\textbf{RNF-5 Seguridad:} los usuarios deben poder identificarse sólidamente con el sistema sin que sus datos o sus credenciales (\textit{tokens}) sean accesibles por terceros, incluso el administrador.
\item
\textbf{RNF-6 Extensibilidad:} la API del sistema debe ser fácilmente extensible a nuevas funcionalidades incorporando de manera eficaz soporte a nuevas peticiones.
\item
\textbf{RNF-7 Persistencia:} los servicios de procesamiento de las camas activas deben mantenerse funcionando aunque no existan clientes activos para evitar retrasos muy altos ante nuevas conexiones.
\item
\textbf{RNF-8 Fiabilidad:} los datos de la aplicación son correctos y actuales además de garantizar una predicción óptima del estado del paciente.
\end{itemize}

\section{Especificación de requisitos}\label{casos-uso}

Los requisitos funcionales generan un conjunto de casos de uso que serán la base del desarrollo de la aplicación. La especificación de los mismos se encuentran entre la tabla~\ref{tabla:tablaCU1} y la tabla~\ref{tabla:tablaCU43}. La representación gráfica se puede ver en los diagramas de las figuras~\ref{fig:cu-1},~\ref{fig:cu-1.1},~\ref{fig:cu-2} y~\ref{fig:cu-2.1}. Existen dos actores, el \textbf{administrador} que se encarga de toda la labor de gestión tanto de usuarios como de camas y el \textbf{usuario} que únicamente puede gestionarse a sí mismo y ver los datos de las camas que tenga permitidas.

\begin{figure}[h]
	\includegraphics[width=\textwidth]{cu-lv1}
	\caption{Diagrama casos de uso - Nivel 0}
	\label{fig:cu-1}
\end{figure}
\begin{figure}[h]
	\includegraphics[width=\textwidth]{cu-lv11}
	\caption{Diagrama CU-2 - Nivel 1}
	\label{fig:cu-1.1}
\end{figure}

\begin{figure}[h]
	\includegraphics[width=\textwidth]{cu-lv12}
	\caption{Diagrama CU-3 - Nivel 1}
	\label{fig:cu-2}
\end{figure}
\begin{figure}[h]
	\includegraphics[width=\textwidth]{cu-lv13}
	\caption{Diagrama CU-4 - Nivel 1}
	\label{fig:cu-2.1}
\end{figure}


\tablaSmallSinColores{Caso de uso 1: Iniciar sesión }{p{0.23\textwidth} p{0.04\textwidth} p{0.64\textwidth}}{tablaCU1}{
	\multicolumn{3}{p{0.68\textwidth}}{CU-1: Iniciar sesión} \\
}
{
	Descripción                            & \multicolumn{2}{p{0.68\textwidth}}{El usuario se identifica en el sistema} \\\hubu
	Precondiciones                         & \multicolumn{2}{p{0.68\textwidth}}{No existe una sesión activa válida} \\\hubu
	Requisitos                         	   & \multicolumn{2}{p{0.68\textwidth}}{RF-1, RF-1.1} \\\hubu
	Usuario                         	   & \multicolumn{2}{p{0.68\textwidth}}{Anónimo} \\\hubu
	\multirow{4}{0.3\textwidth}{Secuencia normal}  & Paso & Acción \\\cline{2-3}
	& 1    & El cliente envía sus credenciales al servidor \\\cline{2-3}
	& 2    & El servidor acepta las credenciales devolviendo el token de sesión \\\hubu
	Postcondiciones                        & \multicolumn{2}{p{0.68\textwidth}}{El usuario tiene una sesión activa válida} \\\hubu
	\multirow{3}{0.3\textwidth}{Excepciones}       & Paso & Acción \\\cline{2-3}
	& 2    & Si las credenciales son incorrectas el servidor responde con error \\\hubu
	Frecuencia                             & Alta \\\hubu
	Importancia                            & Crítico \\\hubu
	Comentarios                            & \multicolumn{2}{p{0.68\textwidth}}{Es siempre lo primero que aparecerá} \\
}

\tablaSmallSinColores{Caso de uso 2: Visualizar de datos }{p{0.23\textwidth} p{0.04\textwidth} p{0.64\textwidth}}{tablaCU2}{
	\multicolumn{3}{p{0.68\textwidth}}{CU-2: Visualizar de datos} \\
}
{
	Descripción                            & \multicolumn{2}{p{0.68\textwidth}}{Ver lista de las camas disponibles} \\\hubu
	Precondiciones                         & \multicolumn{2}{p{0.68\textwidth}}{Sesión activa válida} \\\hubu
	Requisitos                         	   & \multicolumn{2}{p{0.68\textwidth}}{RF-1.3, RF-1.4} \\\hubu
	Usuario                         	   & \multicolumn{2}{p{0.68\textwidth}}{Administrador y Usuario} \\\hubu
	\multirow{2}{0.23\textwidth}{Secuencia normal}  & Paso & Acción \\\cline{2-3}
	& 1    & El cliente solicita ver las camas disponibles \\\hubu
	Postcondiciones                        & \multicolumn{2}{p{0.68\textwidth}}{El cliente está en la pantalla de camas disponibles} \\\hubu
	Frecuencia                             & Alta \\\hubu
	Importancia                            & Alta \\
}

\tablaSmallSinColores{Caso de uso 2.1: Elegir cama }{p{0.23\textwidth} p{0.04\textwidth} p{0.64\textwidth}}{tablaCU21}{
	\multicolumn{3}{p{0.68\textwidth}}{CU-2.1: Elegir cama} \\
}
{
	Descripción                            & \multicolumn{2}{p{0.68\textwidth}}{Elegir cama} \\\hubu
	Precondiciones                         & \multicolumn{2}{p{0.68\textwidth}}{Sesión activa válida} \\\hubu
	Requisitos                         	   & \multicolumn{2}{p{0.68\textwidth}}{RF-1.3, RF-1.4, RF-4} \\\hubu
	Usuario                         	   & \multicolumn{2}{p{0.68\textwidth}}{Logueado} \\\hubu
	\multirow{5}{0.23\textwidth}{Secuencia normal}  & Paso & Acción \\\cline{2-3}
	& 1    & El cliente solicita ver las camas disponibles \\\cline{2-3}
	& 2    & El servidor abre conexiones paralelas para actualizar en tiempo real el estado de las camas \\\cline{2-3}
	& 3    & El cliente decide que cama ver \\\hubu
	\multirow{2}{0.23\textwidth}{Postcondiciones}  & \multicolumn{2}{p{0.68\textwidth}}{El cliente entra en la ventana de los datos en tiempo real} \\\hubu
	Frecuencia                             & Alta \\\hubu
	Importancia                            & Alta \\
}

\tablaSmallSinColores{Caso de uso 2.2: Ver datos en tiempo real }{p{0.23\textwidth} p{0.04\textwidth} p{0.64\textwidth}}{tablaCU22}{
	\multicolumn{3}{p{0.68\textwidth}}{CU-2.2: Ver datos en tiempo real} \\
}
{
	Descripción                            & \multicolumn{2}{p{0.68\textwidth}}{Ver datos en tiempo real} \\\hubu
	Precondiciones                         & \multicolumn{2}{p{0.68\textwidth}}{Sesión activa válida y cama existente y accesible} \\\hubu
	Requisitos                         	   & \multicolumn{2}{p{0.68\textwidth}}{RF-1.3, RF-1.4, RF-4} \\\hubu
	Usuario                         	   & \multicolumn{2}{p{0.68\textwidth}}{Administrador y usuario} \\\hubu
	\multirow{5}{0.23\textwidth}{Secuencia normal}  & Paso & Acción \\\cline{2-3}
	& 1    & El cliente solicita una nueva conexión \\\cline{2-3}
	& 2    & El servidor provee una conexión en tiempo real con los datos \\\hubu
	\multirow{2}{0.23\textwidth}{Postcondiciones}  & \multicolumn{2}{p{0.68\textwidth}}{El usuario tiene una conexión paralela abierta con los datos en tiempo real} \\\hubu
	\multirow{3}{0.23\textwidth}{Excepciones}       & Paso & Acción \\\cline{2-3}
	& 2    & Si un paquete faltase o la señal fuera, débil se alertaría al usuario \\\hubu
	Frecuencia                             & Alta \\\hubu
	Importancia                            & Máxima \\}

\tablaSmallSinColores{Caso de uso 3: Administrar de usuarios }{p{0.23\textwidth} p{0.04\textwidth} p{0.64\textwidth}}{tablaCU3}{
	\multicolumn{3}{p{0.68\textwidth}}{CU-3: Administrar de usuarios} \\
}
{
	Descripción                            & \multicolumn{2}{p{0.68\textwidth}}{Administración de usuario: alta, baja y modificación} \\\hubu
	Precondiciones                         & \multicolumn{2}{p{0.68\textwidth}}{Sesión de administrador válida} \\\hubu
	Requisitos                         	   & \multicolumn{2}{p{0.68\textwidth}}{RF-3} \\\hubu
	Usuario                         	   & \multicolumn{2}{p{0.68\textwidth}}{Administrador} \\\hubu
	\multirow{3}{0.23\textwidth}{Secuencia normal}  & Paso & Acción \\\cline{2-3}
	& 1    & El administrador entra en el menú de administración de usuarios \\\hubu
	\multirow{2}{0.23\textwidth}{Postcondiciones}  & \multicolumn{2}{p{0.68\textwidth}}{El administrador está en el menú de administración de usuarios} \\\hubu
	Frecuencia                             & Baja \\\hubu
	Importancia                            & Alta \\
}

\tablaSmallSinColores{Caso de uso 3.1: Añadir usuarios }{p{0.23\textwidth} p{0.04\textwidth} p{0.64\textwidth}}{tablaCU31}{
	\multicolumn{3}{p{0.68\textwidth}}{CU-3.1: Añadir usuarios} \\
}
{
	Descripción                            & \multicolumn{2}{p{0.68\textwidth}}{Añadir usuarios} \\\hubu
	Precondiciones                         & \multicolumn{2}{p{0.68\textwidth}}{Sesión de administración activa} \\\hubu
	Requisitos                         	   & \multicolumn{2}{p{0.68\textwidth}}{RF-3.1} \\\hubu
	Usuario                         	   & \multicolumn{2}{p{0.68\textwidth}}{Administrador} \\\hubu
	\multirow{6}{0.23\textwidth}{Secuencia normal}  & Paso & Acción \\\cline{2-3}
	& 1    & El administrador elige añadir un nuevo usuario \\\cline{2-3}
	& 2    & Se introduce un nombre de usuario para identificarlo \\\cline{2-3}
	& 3    & Se introduce una contraseña dos veces \\\cline{2-3}
	& 4    & Se almacenan los datos \\\hubu
	Postcondiciones                        & \multicolumn{2}{p{0.68\textwidth}}{Existe un nuevo usuario en el sistema} \\\hubu
	\multirow{4}{0.23\textwidth}{Excepciones}       & Paso & Acción \\\cline{2-3}
	& 2    & Si el nickname existiese \\\cline{2-3}
	& 3    & La contraseña añadida no coincide en las dos ocasiones \\\hubu
	Frecuencia                             & Baja \\\hubu
	Importancia                            & Alta \\
}

\tablaSmallSinColores{Caso de uso 3.2: Modificar contraseña }{p{0.23\textwidth} p{0.04\textwidth} p{0.64\textwidth}}{tablaCU32}{
	\multicolumn{3}{p{0.68\textwidth}}{CU-3.2: Modificar contraseña} \\
}
{
	Descripción                            & \multicolumn{2}{p{0.68\textwidth}}{Cambiar la contraseña de un usuario} \\\hubu
	Precondiciones                         & \multicolumn{2}{p{0.68\textwidth}}{Sesión activa válida, usuario existente} \\\hubu
	Requisitos                         	   & \multicolumn{2}{p{0.68\textwidth}}{RF-3.2} \\\hubu
	Usuario                         	   & \multicolumn{2}{p{0.68\textwidth}}{Administrador y Usuario} \\\hubu
	\multirow{6}{0.23\textwidth}{Secuencia normal}  & Paso & Acción \\\cline{2-3}
	& 1    & Si es usuario normal ir a 3 \\\cline{2-3}
	& 2    & Si es administrador elegir a qué usuario cambiar la contraseña \\\cline{2-3}
	& 3    & Se introduce una contraseña nueva dos veces \\\cline{2-3}
	& 4    & Se actualizan los datos \\\hubu
	Postcondiciones                        & \multicolumn{2}{p{0.68\textwidth}}{La contraseña ha cambiado} \\\hubu
	\multirow{3}{0.23\textwidth}{Excepciones}       & Paso & Acción \\\cline{2-3}
	& 3    & La contraseña añadida no coincide en las dos ocasiones \\\hubu
	Frecuencia                             & Baja \\\hubu
	Importancia                            & Alta \\
}

\tablaSmallSinColores{Caso de uso 3.3: Borrar usuario }{p{0.23\textwidth} p{0.04\textwidth} p{0.64\textwidth}}{tablaCU33}{
	\multicolumn{3}{p{0.68\textwidth}}{CU-3.3: Borrar usuario} \\
}
{
	Descripción                            & \multicolumn{2}{p{0.68\textwidth}}{Elimina un usuario de la base de datos} \\\hubu
	Precondiciones                         & \multicolumn{2}{p{0.68\textwidth}}{Sesión de administración válida, usuario existente} \\\hubu
	Requisitos                         	   & \multicolumn{2}{p{0.68\textwidth}}{RF-3.3} \\\hubu
	Usuario                         	   & \multicolumn{2}{p{0.68\textwidth}}{Administrador} \\\hubu
	\multirow{3}{0.23\textwidth}{Secuencia normal}  & Paso & Acción \\\cline{2-3}
	& 1    & Elegir a que usuario (no administrador) eliminar \\\cline{2-3}
	& 2    & Eliminar usuario y todos los datos vinculados \\\hubu
	Postcondiciones                        & \multicolumn{2}{p{0.68\textwidth}}{El usuario ha sido eliminado} \\\hubu
	Frecuencia                             & Baja \\\hubu
	Importancia                            & Media \\
}

\tablaSmallSinColores{Caso de uso 4: Administrar de camas }{p{0.23\textwidth} p{0.04\textwidth} p{0.64\textwidth}}{tablaCU4}{
	\multicolumn{3}{p{0.68\textwidth}}{CU-4: Administrar de camas} \\
}
{
	Descripción                            & \multicolumn{2}{p{0.68\textwidth}}{Administración de camas: alta, baja, modificación y asignación a usuarios} \\\hubu
	Precondiciones                         & \multicolumn{2}{p{0.68\textwidth}}{Sesión de administración válida} \\\hubu
	Requisitos                         	   & \multicolumn{2}{p{0.68\textwidth}}{RF-2} \\\hubu
	Usuario                         	   & \multicolumn{2}{p{0.68\textwidth}}{Administrador} \\\hubu
	\multirow{3}{0.23\textwidth}{Secuencia normal}  & Paso & Acción \\\cline{2-3}
	& 1    & El administrador entra en el menú de administración de camas \\\hubu
	\multirow{2}{0.23\textwidth}{Postcondiciones}  & \multicolumn{2}{p{0.68\textwidth}}{El administrador está en el menú de administración de camas} \\\hubu
	Frecuencia                             & Baja \\\hubu
	Importancia                            & Media \\
}

\tablaSmallSinColores{Caso de uso 4.1: Añadir cama }{p{0.23\textwidth} p{0.04\textwidth} p{0.64\textwidth}}{tablaCU41}{
	\multicolumn{3}{p{0.68\textwidth}}{CU-4.1: Añadir cama} \\
}
{
	Descripción                            & \multicolumn{2}{p{0.68\textwidth}}{Añadir cama} \\\hubu
	Precondiciones                         & \multicolumn{2}{p{0.68\textwidth}}{Sesión de administración válida} \\\hubu
	Requisitos                         	   & \multicolumn{2}{p{0.68\textwidth}}{RF-2.1} \\\hubu
	Usuario                         	   & \multicolumn{2}{p{0.68\textwidth}}{Administrador} \\\hubu
	\multirow{6}{0.23\textwidth}{Secuencia normal}  & Paso & Acción \\\cline{2-3}
	& 1    & El administrador elige añadir una nueva cama \\\cline{2-3}
	& 2    & Se introduce el grupo multicast de la cama (IP y Puerto) \\\cline{2-3}
	& 3    & Se introduce el nombre identificador\\\cline{2-3}
	& 4    & Se almacenan los datos \\\hubu
	Postcondiciones                        & \multicolumn{2}{p{0.68\textwidth}}{Existe una nueva cama en el sistema} \\\hubu
	\multirow{3}{0.23\textwidth}{Excepciones}       & Paso & Acción \\\cline{2-3}
	& 2    & El grupo multicast pertenece a otra cama \\\cline{2-3}
	& 3    & El nombre identificativo existe para otra cama \\\hubu
	Frecuencia                             & Media \\\hubu
	Importancia                            & Crítica \\\hubu
	\multirow{3}{0.23\textwidth}{Comentarios} & \multicolumn{2}{p{0.68\textwidth}}{El grupo multicast se configura en la cama y el administrador solamente debe conocerlo, no configurar la cama física} \\
}

\tablaSmallSinColores{Caso de uso 4.2: Modificar cama }{p{0.23\textwidth} p{0.04\textwidth} p{0.64\textwidth}}{tablaCU42}{
	\multicolumn{3}{p{0.68\textwidth}}{CU-4.2: Modificar cama} \\
}
{
	Descripción                            & \multicolumn{2}{p{0.68\textwidth}}{Modificar los datos de la cama} \\\hubu
	Precondiciones                         & \multicolumn{2}{p{0.68\textwidth}}{Sesión de administración válida, cama existente} \\\hubu
	Requisitos                         	   & \multicolumn{2}{p{0.68\textwidth}}{RF-2.2} \\\hubu
	Usuario                         	   & \multicolumn{2}{p{0.68\textwidth}}{Administrador} \\\hubu
	\multirow{5}{0.23\textwidth}{Secuencia normal}  & Paso & Acción \\\cline{2-3}
	& 1    & Se elige que cama modificar \\\cline{2-3}
	& 2    & Se actualizan los datos a conveniencia del administrador según CU-4.1\\\cline{2-3}
	& 4    & Se actualizan los datos \\\hubu
	Postcondiciones                        & \multicolumn{2}{p{0.68\textwidth}}{Los datos de la cama se modifican} \\\hubu
	\multirow{2}{0.23\textwidth}{Excepciones}       & Paso & Acción \\\cline{2-3}
	& 2    & Mismas excepciones que en CU-4.1 \\\hubu
	Frecuencia                             & Baja \\\hubu
	Importancia                            & Alta \\
}

\tablaSmallSinColores{Caso de uso 4.3: Borrar cama }{p{0.23\textwidth} p{0.04\textwidth} p{0.64\textwidth}}{tablaCU43}{
	\multicolumn{3}{p{0.68\textwidth}}{CU-4.3: Borrar cama} \\
}
{
	Descripción                            & \multicolumn{2}{p{0.68\textwidth}}{Elimina una cama de la base de datos} \\\hubu
	Precondiciones                         & \multicolumn{2}{p{0.68\textwidth}}{Sesión de administrador válida, cama existente} \\\hubu
	Requisitos                         	   & \multicolumn{2}{p{0.68\textwidth}}{RF-2.3} \\\hubu
	Usuario                         	   & \multicolumn{2}{p{0.68\textwidth}}{Administrador} \\\hubu
	\multirow{3}{0.23\textwidth}{Secuencia normal}  & Paso & Acción \\\cline{2-3}
	& 1    & Elegir a que cama eliminar \\\cline{2-3}
	& 2    & Eliminar cama y todos los datos vinculados \\\hubu
	Postcondiciones                        & \multicolumn{2}{p{0.68\textwidth}}{La cama ya no está en la base de datos} \\\hubu
	Frecuencia                             & Baja \\\hubu
	Importancia                            & Media \\
}

\tablaSmallSinColores{Caso de uso 4.4: Asignar cama a usuario }{p{0.23\textwidth} p{0.04\textwidth} p{0.64\textwidth}}{tablaCU44}{
	\multicolumn{3}{p{0.68\textwidth}}{CU-4.4: Asignar cama a usuario} \\
}
{
	Descripción                            & \multicolumn{2}{p{0.68\textwidth}}{Permite a un usuario ver los datos de una cama o quitar ese permiso} \\\hubu
	Precondiciones                         & \multicolumn{2}{p{0.68\textwidth}}{Sesión de administración válida, cama y usuario existentes} \\\hubu
	Requisitos                         	   & \multicolumn{2}{p{0.68\textwidth}}{RF-2.4} \\\hubu
	Usuario                         	   & \multicolumn{2}{p{0.68\textwidth}}{Administrador} \\\hubu
	\multirow{5}{0.23\textwidth}{Secuencia normal}  & Paso & Acción \\\cline{2-3}
	& 1    & Elegir cama \\\cline{2-3}
	& 2    & Elegir usuario \\\cline{2-3}
	& 3    & Si la relación existe se puede eliminar el permiso \\\cline{2-3}
	& 3    & Si la relación no existe se puede crear el permiso \\\hubu
	Postcondiciones                        & \multicolumn{2}{p{0.68\textwidth}}{El usuario tiene acceso a la cama, o pierde el mismo} \\\hubu
	Frecuencia                             & Media \\\hubu
	Importancia                            & Crítica \\
}