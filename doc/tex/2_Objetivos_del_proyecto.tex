\capitulo{2}{Objetivos del proyecto}

\section{Objetivos generales}

\begin{itemize}
	\item Estudio del estado en materia de detección de crisis epilépticas, tanto en \textit{hardware}, datos utilizados y técnicas y modelos existentes desarrollados por otros científicos. Con esto se busca explorar técnicas no antes utilizadas como optimizar esfuerzos por los métodos que ya han probado su utilidad.
	\item Exploración e interpretación de los datos, las formas por las cuales se pueden representar y como se distribuyen los distintos ejemplos que representan crisis respecto a las situaciones normales. Aplicación de filtros, análisis estadísticos, proyecciones de n-variedad.
	\item Exploración de técnicas de balanceo de datos existentes y aprendizaje automático para conjuntos de datos desequilibrados. Buscar la mejor combinación de técnicas de balanceo y modelos de aprendizaje automático para optimizar la precisión.
	\item Búsqueda de un modelo que haga una clasificación lo más correcta posible centrando que la predicción sea acertada en situaciones de crisis mediante la optimización del valor del ratio de verdaderos positivos.
	\item Creación de una \textit{API REST} con la que poder distribuir en tiempo real los datos de cama y sus predicciones.
\end{itemize}

\section{Objetivos técnicos}

\begin{itemize}
	\item Hacer uso de las herramientas de minería de datos de \textit{sci-kit learn} y \textit{Weka}.
	\item Crear una serie de transformadores de \textit{scikit-learn} de datos para facilitar el preprocesado.
	\item Desarrollar una \textit{API REST} sencilla y fácil de usar.
	\item Crear una interfaz web que permita ver los datos en tiempo real de los pacientes con los menores márgenes temporales posibles.
\end{itemize}

\section{Objetivos personales}

\begin{itemize}
	\item Contribuir a la mejora de la calidad de vida de pacientes que sufran de epilepsia.
	\item Profundizar en el trabajo de investigador, sus metodologías y las fases de por las que pasa una investigación.
	\item Comprender más técnicas de minería de datos, nuevos modelos y nuevas formas de abordar análisis de datos.
	\item Completar mi formación académica con el desarrollo de una aplicación que englobe la mayor cantidad del conocimiento adquirido en el estudio del grado.
\end{itemize}