\capitulo{4}{Técnicas y herramientas}

Para comenzar a comprender mejor los datos los primeros pasos realizados han sido la disminución de dimensionalidad de los datos. Entre las técnicas que se utilizaron están el análisis de componentes principales~\cite{wiki:pca} y otras proyecciones de variedad (\textit{manifolds})~\cite{wiki:manifold}, en particular las proyecciones bidimensionales utilizando \textit{Spectral Embedding} y \textit{t-SNE}. Todos estos tienen la característica que son no supervisados por lo que nos permiten ver si las situaciones de crisis y las normales son fácilmente separables. Para esto utilizamos las funciones de \textit{scikit-learn}~\cite{tool:scikit-learn} y \textit{tsnecuda}~\cite{tool:tsnecuda}. Sin embargo debido al tamaño de los datos con las proyecciones a 2-variedad solamente pudimos utilizar un día como referencia cada vez y la conclusión directa viendo las proyecciones brutas es que no son separables. Sin embargo si se puede ver que gran parte de las situaciones de crisis se concentran en las proyecciones de PCA y \textit{Spectral Embedding}.


